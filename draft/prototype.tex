\section{Prototyping}
\label{sec:prototype}

In this paper, we provide a prototyping of ICT platforms for mobility
CPS applications, particularly taking an example of autonomous driving.
We apply the cloud computing paradigm for autonomous driving to mitigate
high computational cost imposed on wimpy embedded vehicular systems.
One of the major goals of this cloud-based autonomous driving system is
to offload the computation requirement of autonomic control and
environmental perception to a remote HPC server from a local vehicular
system.
Note that we focus on ICT platforms.
Implementations of autonomic control and environmental perception for
autonomous driving are outside the scope of this paper.
Interested readers are encouraged to refer to our different
contributions \cite{Hirabayashi13, Kagami13}.

In the rest of this section, we present two ICT platforms for the
development of cloud-based autonomous driving.
The first platform is a smartphone application that controls the vehicle
from remote sources.
The second platform is also a smartphone application that transfers
captured images to the HPC server as fast as possible.

\subsection{Remote Vehicle Control}

\begin{figure}[!t]
 \centering
 \includegraphics[width=0.6\hsize]{fig/Andrive.pdf}
 \caption{A smartphone application for remote vehicle control.}
 \label{fig:andrive}
\end{figure}

We have a TOYOTA Prius that is modified to be able to overwrite the
control of steering, accel, and break from the computer.
This computer called ``local master computer'' is connected to an
additional embedded board that can directly send signals to the inside
wire system to control the vehicle.
We omit a detailed description of this autonomous driving system, as the
primary focus of this paper is the measurement of overhead and latency.

Fig. \ref{fig:andrive} illustrates the conceptual architecture of our
experimental remote vehicle control system.
We use a commodity smartphone to send control commands to the local
master computer equipped within the vehicle through the network.
The smartphone application determines the steering angle from the gyro
sensor while the accel and the break strokes are controlled by the
graphics user interface.
Thus, we can intuitively use a smartphone as if it were a game
controller of the vehicle remotely.

\subsection{Networked Image Processing}

\begin{figure}[!t]
 \centering
 \includegraphics[width=\hsize]{fig/TIPIC.pdf}
 \caption{Networked image processing.}
 \label{fig:tipic}
\end{figure}

The computional cost of environmental perception in mobility CPS is
highly expensive.
The perception algorithm is often computationally complex and the volume
of input data from laser sensors and cameras is not afforable for mobile
embedded systems.
Therefore it would be significant if we could offload these compute
tasks to the cloud.

Fig. \ref{fig:tipic} illustrates the conceptual architecture of our
experimental networked image processing system.
We capture images in real-time from commodity smartphones attached in
the vehicle and transfer them over the network to the HPC server where
the actual image processing tasks are executing.
The results of image processing such as surrounding vehicle and
pedestrian detection are fed back to the vehicular system.
Though we have implemented many variants of image processing algorithms
\cite{Hirabayashi13}, they are not focused on in this paper.
Instead we investigate if current commodity ICT platforms can transfer
data over the network while meeting the desired throughput and latency.

