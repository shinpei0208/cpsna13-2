\section{Conclusion}
\label{sec:conclusion}

This paper has presented a prototyping of commodity ICT platforms using
smartphones and PC servers for a cloud-based autonomous vehicle, which
is a representative of mobility CPS applications in the current state of
the art.
We demonstrated that commodity WiFi could provide a feedback-control
period of $5ms$ to $10ms$ for remote autonomic control while a frame
rate of $10$fps to $20$fps for networked image processing.
Using commodity LTE, we observed some performance loss from WiFi
particularly in latency but it may become useful in a decade as
technology advances read today.
To the best of our knowledge, this is the first quantitative evidence
that explains the availability of commodity ICT platforms for mobility
CPS applications.

The tools presented in this paper are all open-source, and may be
downloaded from \url{https://github.com/cs005/}.

In future work, we plan to prototype a complete cloud-based autonomous
vehicle and investigate the availability of commodity ICT platforms
using a real case study.
We also seek for offloading path planning and/or mission planning tasks
to the cloud, given that they could also be computationally expensive as
reported in \cite{McNaughton_ICRA11}.
Further grand challenges include a generalization of ``mobility CPS in
the cloud'' beyond a particular case study of autonomous driving.
It is a frontier of mobility CPS to answer the question of what tasks
should remain in the mobility agent itself.
Since recent powerful compute devices are becoming more suitable for
low-latency real-time computing \cite{Kato13}, it is important to
provide a design nob of hierarchical architectures from the on-board
mobility agent through the cloud environment.