\begin{abstract}
 The next-generation mobility infrastructure requires computer systems
 to be more and more intelligent and interactive with the physical
 world.
 A grand challenge to this cyber-physical systems (CPS) problem can be
 found in real-time computing of autonomic control and environmental
 perception.
 Given the volume of information in the physical world, the
 computational cost of autonomic control and environmental perception
 is likely very high, which may not be affordable for embedded mobile
 devices.
 In this paper, we explore a possibility of leveraging cloud computing
 technology to support such computationally expensive operations of
 intelligent mobility CPS applications.
 Specifically we quantify the overhead of offloading computations to the
 cloud using commodity information and communication technology (ICT)
 platforms taking an example of autonomous driving.
 This quantification of overhead plays a vital role in the design issue
 of emerging mobility CPS applications.
 Our experimental results demonstrate that the current communication
 standards including WiFi and LTE achieve sufficient throughput and
 latency even when used with mobile smartphones to transfer large images
 and high-rate control commands over the network.
\end{abstract}