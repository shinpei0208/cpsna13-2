\section{Introduction}
\label{sec:introduction}

Mobility is an essential piece of our life.
In recent years, transportation systems and automobiles are becoming
more and more intelligent underlying high-efficient mobility of the
societal infrastructure.
Such innovations in mobility will reduce car accidents, remove human\'s
stress, improve traffic throughput, and create new markets.
To this end, the next-generation mobility technology is expected to
achieve a tight coordination of computer systems and physical elements,
often referred to as cyber-physical systems (CPS), facilitating a
cyber-understanding of the physical world.
A core challenge to this mobility CPS, however, is a power and space
constraint of each mobility component.
For example, moving vehicles cannot accommodate rich computer systems
due to the limited battery power.
Particularly the computational cost with respect to understanding the
physical world is very high and the computing capability of current
vehicular systems is woefully inadequate.
In order to address this trade-off, mobility CPS must seek for a
cooperative solution with advanced information and communication
technology (ICT) such as cloud computing.

A good example of compute-intensive mobility CPS applications is an
autonomous vehicle~\cite{Guizzo11, Levinson11, Urmson08}.
It must recognize roads, traffic signs, surrounding vehicles, and
pedestrians in real-time.
An autonomous vehicle in the current state of the art tends to use laser
sensors and/or cameras for those environmental perception tasks.
The laser sensors can detect object edges as a set of 3-D points by
hardware, reducing the computational requirement imposed on the
vehicular system software, but are generally very expensive way beyond
consumer electronics prices.
On the other hand, the cameras are less expensive in price but are
available at the expense of computational cost, because image processing
is highly compute-intensive.
It would require a rich set of multicore CPUs and hardware accelerators
such as GPUs to meet the desired frame rate.
Unfortunately these devices may not be affordable for battery-operated
vehicular systems due to power consumption issues.
As mentioned earlier, therefore, we should seek for a possibility of
leveraging cloud computing technology to offload compute-intensive tasks
onto high-performance computing (HPC) servers over the network.
A question raised herein is ``what is the overhead of offloading
computation and associated data to the cloud?''
If this overhead is acceptable, commodity ICT platforms will be a strong
basis for mobility CPS applications.

\textbf{Contribution:}
We present a prototype implementation of commodity ICT platforms for
mobility CPS applications to quantify the overhead of offloading
computation and data to the cloud.
Specifically we use a smartphone as an example of commodity ICT
platforms to capture images and control an autonomous vehicle as an
example of mobility CPS applications.
The overhead of transferring images over the network governs the frame
rate of image processing in the cloud while that of transferring control
commands determines the minimum feedback-control period of autonomous
driving.
We demonstrate that the bottleneck of networked image processing can be
found in the computation time itself rather than the network
communication overhead.
We also find that the average network throughput of commodity ICT is
sufficient to execute autonomic control but the worst-case latency must
be bounded to provide stability.
This is a useful insight into a coordination of commodity ICT and
mobility CPS.

\textbf{Organization:}
The rest of this paper is organized as follows.
Section \ref{sec:concept} describes the basic concept behind this
paper.
Section \ref{sec:prototype} presents our prototyping of ICT platforms
for an autonomous vehicle as an example of mobility CPS applications.
Section \ref{sec:evaluation} provides the evaluation of overhead imposed
on data transfers over the network relevant to remote vehicle control
and networked image processing.
This paper concludes in Section \ref{sec:conclusion}.